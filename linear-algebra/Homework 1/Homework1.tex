\documentclass[12pt, a4paper]{article}
\usepackage{amsmath}
\usepackage{amssymb}

\begin{titlepage}
\title{
    \textbf{Homework 1} \\ \vspace{1cm}
}

\author{
\vspace*{0.5cm}
    David Abeiku Saah \\ \vspace{0.4cm}
    MATH212: Linear Algebra \\ \vspace{0.5cm}
    Dr Ayawoa Dagbovie \\
}

\date{August 31, 2023}
\end{titlepage}

\begin{document}

\maketitle

\newpage

\section*{Problem 1}

% 4 x 6 matrix
Finding the general solution of the system whose augmented matrix is:
\[
\begin{bmatrix}
    1 & 0 & -5 & 0 & -8 & 3 \\
    0 & 1 & 4 & -1 & 0 & 6 \\
    0 & 0 & 0 & 0 & 1 & 0 \\
    0 & 0 & 0 & 0 & 0 & 0
\end{bmatrix}
\]

Converting to a reduced row echelon form:

\[R_1 \rightarrow 8R_3 + R_1\]

\[
= \begin{bmatrix}
    1 & 0 & -5 & 0 & 0 & 3 \\
    0 & 1 & 4 & -1 & 0 & 6 \\
    0 & 0 & 0 & 0 & 1 & 0 \\
    0 & 0 & 0 & 0 & 0 & 0
\end{bmatrix}
\]
There are 5 variables because the augmented matrix has 6 columns.
\linebreak
Associated system of equations:
\begin{align*}
    x_1 - 5x_3 &= 3 \\
    x_2 + 4x_3 - x_4 &= 6 \\
    x_5 &= 0 \\
    0 &= 0 \\
\end{align*}
\[
\therefore \text{The general solution of the system is: }
\begin{cases}
    x_1 = 3 + 5x_3  \\
    x_2 = 6 - 4x_3 + x_4 \\
    x_3 \text{ is free} \\
    x_4 \text{ is free} \\
    x_5 = 0 \\
    x_6 \text{ is free}
\end{cases}
\]

\pagebreak

\section*{Problem 2}

System of equations:
\begin{align*}
    x_1 - 3x_2 = 1 \\
    2x_1 + hx_2 = k \\
\end{align*}
Augmented Matrix:
\[
\begin{bmatrix}
    1 & -3 & 1 \\
    2 & h & k \\
\end{bmatrix}
\]
\linebreak
Finding the echelon form of the augmented matrix:
\[R_2 \rightarrow -2R_1 + R_2\]
\[
= \begin{bmatrix}
    1 & -3 & 1 \\
    0 & h + 6 & k - 2 \\
\end{bmatrix}
\]

% add subsections
\subsection*{(a)}
The system has no solution when any of its equations results in an indeterminate form. This occurs when $h + 6 = 0$ and $k - 2 \neq 0$.
\[
\therefore \text{The system has no solution when } h = -6 \text{ and } k \in \mathbb{R}, k \neq 2.
\]

\subsection*{(b)}
The system has a unique solution when all the columns (except the rightmost) are pivot columns. This occurs when $h+6 \neq 0$ and $k - 2$ can result to any real number.
\[
\therefore \text{The system has a unique solution when } h \neq -6 \text{ and } k \in \mathbb{R}.
\]

\subsection*{(c)}
The system has infinitely many solutions when there is at least one free variable. This occurs when $h + 6 = 0$ and $k - 2 = 0$.
\[
\therefore \text{The system has infinitely many solutions when } h = -6 \text{ and } k = 2.
\]

\end{document}