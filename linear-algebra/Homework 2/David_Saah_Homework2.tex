
\documentclass[12pt, a4paper]{article}
\usepackage{amsmath}
\usepackage{amssymb}

\begin{titlepage}
\title{
    \textbf{Homework 2} \\ \vspace{1cm}
}

\author{
\vspace*{0.5cm}
    David Abeiku Saah \\ \vspace{0.4cm}
    MATH212: Linear Algebra \\ \vspace{0.5cm}
    Dr Ayawoa Dagbovie \\
}

\date{September 7, 2023}
\end{titlepage}

\begin{document}

\maketitle

\newpage

\section*{Problem 1}
\[
\boldsymbol{a_1}  = \begin{bmatrix}
        1 \\
        0 \\
        1
    \end{bmatrix}, \quad
\boldsymbol{a_2}  = \begin{bmatrix}
       -2 \\
        3 \\
       -2
\end{bmatrix}, \quad
\boldsymbol{a_3}  = \begin{bmatrix}
       -6 \\
        7 \\
        5
\end{bmatrix}, \quad
\boldsymbol{b}  = \begin{bmatrix}
        11 \\
       -5 \\
        9
\end{bmatrix}
\]
For \textbf{b} to be a linear combination of $\boldsymbol{a_1}$, $\boldsymbol{a_2}$, and $\boldsymbol{a_3}$, the augmented matrix, $\begin{bmatrix} \boldsymbol{a_1} \quad  \boldsymbol{a_2} \quad \boldsymbol{a_3} \quad \boldsymbol{b} \end{bmatrix}$ must be consistent.

Augmented matrix:
\[
\begin{bmatrix}
    1 & -2 & -6 & 11 \\
    0 & 3 & 7 & -5 \\
    1 & -2 & 5 & 9
\end{bmatrix}
\]

\[R_3 \rightarrow R_3 - R_1\]

\[
\begin{bmatrix}
    1 & -2 & -6 & 11 \\
    0 & 3 & 7 & -5 \\
    0 & 0 & 11 & -2
\end{bmatrix}
\]

The matrix in echelon form above is consistent because all columns except the rightmost one (the last column) are pivot columns.

\[
\therefore \boldsymbol{b} \text{ is a linear combination of } \boldsymbol{a_1}, \boldsymbol{a_2}, \text{ and } \boldsymbol{a_3}.
\]
\newpage
\section*{Problem 2}

\[
\boldsymbol{v_1}  = \begin{bmatrix}
        50 \\
        56 \\
        34
    \end{bmatrix}, \quad
\boldsymbol{v_2}  = \begin{bmatrix}
       35 \\
        14 \\
       123
\end{bmatrix}
\]

\subsection*{(a)}
$3\boldsymbol{v_2}$ represents the output of Farm $\#2$ after operating for 3 months.

\subsection*{(b)}
\[
    \text{Let } \boldsymbol{b} = \text{target output}  = \begin{bmatrix}
        830 \\
        728 \\
        1358
    \end{bmatrix}
\]
Vector equation whose solution gives the number of months each farm should operate in order to produce the target output:
\[
    x_1\boldsymbol{v_1} + x_2\boldsymbol{v_2} = \boldsymbol{b}
\]

\[
    \implies x_1\begin{bmatrix}
        50 \\
        56 \\
        34
    \end{bmatrix} + x_2\begin{bmatrix}
         35 \\
          14 \\
         123
    \end{bmatrix} = \begin{bmatrix}
        830 \\
        728 \\
        1358
    \end{bmatrix}
\]
\newpage
\section*{Problem 3}
\[
    A = \begin{bmatrix}
        1 & -2 & -1 \\
       -2 & 2 & 0 \\
        4 & -1 & 3
\end{bmatrix}, \quad
    \boldsymbol{b} = \begin{bmatrix}
        b_1 \\
        b_2 \\
        b_3
\end{bmatrix}
\]
Augmented matrix for $A\boldsymbol{x} = \boldsymbol{b}$:
\[
    \begin{bmatrix}
        1 & -2 & -1 & b_1 \\
       -2 & 2 & 0 & b_2 \\
        4 & -1 & 3 & b_3
\end{bmatrix}
\]

\[R_2 \rightarrow R_2 + 2R_1\]

\[
    \begin{bmatrix}
        1 & -2 & -1 & b_1 \\
       0 & -2 & -2 & 2b_1+b_2 \\
        4 & -1 & 3 & b_3
\end{bmatrix}
\]

\[R_3 \rightarrow R_3 - 4R_1\]

\[
    \begin{bmatrix}
        1 & -2 & -1 & b_1 \\
       0 & -2 & -2 & 2b_1+b_2 \\
        0 & 7 & 7 & b_3-4b_1
\end{bmatrix}
\]

\[R_3 \rightarrow R_3 + \frac{7}{2}R_2\]

\[
    \begin{bmatrix}
        1 & -2 & -1 & b_1 \\
       0 & -2 & -2 & 2b_1+b_2 \\
        0 & 0 & 0 & b_3-4b_1+\frac{7}{2}(2b_1+b_2)
\end{bmatrix}
\]

Simplifying:
\[
    \begin{bmatrix}
        1 & -2 & -1 & b_1 \\
       0 & -2 & -2 & 2b_1+b_2 \\
        0 & 0 & 0 & 3b_1+\frac{7}{2}b_2+b_3
    \end{bmatrix}
\]
\[\]
\subsection*{(a)}
The third entry in column 4 equals $3b_1+\frac{7}{2}b_2+b_3$. The equation, $A\boldsymbol{x} = \boldsymbol{b}$ is not consistent for every $\boldsymbol{b}$ because some choices of $\boldsymbol{b}$ can make $3b_1+\frac{7}{2}b_2+b_3$ nonzero. For example:

If $\boldsymbol{b} = \begin{bmatrix} 1 \\ 1 \\ 1 \end{bmatrix}$, then $3b_1+\frac{7}{2}b_2+b_3 = 3(1)+\frac{7}{2}(1)+1 = \frac{15}{2} \neq 0$.
\[
    \therefore \text{The equation, } A\boldsymbol{x} = \boldsymbol{b} \text{ does not have a solution for all possible } \boldsymbol{b}.
\]

\subsection*{(b)}
For the equation, $A\boldsymbol{x} = \boldsymbol{b}$ to have a solution, all the entries of $\boldsymbol{b}$ must satisfy the equation:
\[3b_1+\frac{7}{2}b_2+b_3 = 0\]

$\therefore$ The set of all possible $\boldsymbol{b}$ is the set of all vectors in $\mathbb{R}^3$ that satisfy the equation:
\[3b_1+\frac{7}{2}b_2+b_3 = 0\]

\end{document}