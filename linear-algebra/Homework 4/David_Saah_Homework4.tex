\documentclass[12pt, a4paper]{article}
\usepackage{amsmath}
\usepackage{amssymb}

\begin{titlepage}
\title{
    \textbf{Homework 4} \\ \vspace{1cm}
}

\author{
\vspace*{0.5cm}
    David Abeiku Saah \\ \vspace{0.4cm}
    MATH212: Linear Algebra \\ \vspace{0.5cm}
    Dr Ayawoa Dagbovie \\
}

\date{September 29, 2023}
\end{titlepage}

\begin{document}

\maketitle

\newpage

\section*{Problem 1}

Given matrix:

\[
    A = \begin{bmatrix}
        12 & 10 & -6 & 8 & 4 & -14 \\
        -7 & -6 & 4 & -5 & -7 & 9 \\
        9 & 9 & -9 & 9 & 9 & -18 \\
        -4 & -3 & -1 & 0 & -8 & 1 \\
        8 & 7 & -5 & 6 & 1 & -11 \\
    \end{bmatrix}
\]
The reduced row echelon form of $A$ using software (python) is:
\[
    \begin{bmatrix}
        1 & 0 & 0 & 1 & 0 & 1 \\
        0 & 1 & 0 & -1 & 0 & -2 \\
        0 & 0 & 1 & -1 & 0 & 1 \\
        0 & 0 & 0 & 0 & 1 & 0 \\
        0 & 0 & 0 & 0 & 0 & 0 \\
    \end{bmatrix}
\]
The pivot columns are $1, 2, 3,\text{ and } 5$. Hence, let matrix B = $\begin{bmatrix}
    12 & 10 & -6 & 4 \\
    -7 & -6 & 4 & -7 \\
    9 & 9 & -9 & 9 \\
    -4 & -3 & -1 & -8 \\
    8 & 7 & -5 & 1 \\
\end{bmatrix}$
Solving $B\boldsymbol{x}=\boldsymbol{0}$.

\[
    \begin{bmatrix}
        12 & 10 & -6 & 4 \\
        -7 & -6 & 4 & -7 \\
        9 & 9 & -9 & 9 \\
        -4 & -3 & -1 & -8 \\
        8 & 7 & -5 & 1 \\
    \end{bmatrix} \begin{bmatrix}
        x_1 \\
        x_2 \\
        x_3 \\
        x_4 \\
    \end{bmatrix} = \begin{bmatrix}
        0 \\
        0 \\
        0 \\
        0 \\
    \end{bmatrix}
\]

The row echelon form of $B$ using software (python) is:
\[
    \begin{bmatrix}
        1 & 0 & 0 & 0 \\
        0 & 1 & 0 & 0 \\
        0 & 0 & 1 & 0 \\
        0 & 0 & 0 & 1 \\
        0 & 0 & 0 & 0 \\
    \end{bmatrix}
\]
$\therefore$ $x_1 = 0, x_2 = 0, x_3 = 0, x_4 = 0$.

\newpage

\section*{Problem 2}

\[
    v_1 = \begin{bmatrix}
        1 \\
        -3 \\
        -5 \\
        \end{bmatrix}, v_2 = \begin{bmatrix}
            -3 \\
            9 \\
            15 \\
        \end{bmatrix}, v_3 = \begin{bmatrix}
            2 \\
            -5 \\
            h \\
        \end{bmatrix}
\]

\subsection*{a.}
For $v_3$ to be in Span$\{v_1, v_2\}$, then $v_3$ must be a linear combination of $v_1$ and $v_2$. This is the same as checking for consistency of the matrix equation $A\boldsymbol{x} = \boldsymbol{b}$ where $A = \begin{bmatrix}
    1 & -3 \\
    -3 & 9 \\
    -5 & 15 \\
\end{bmatrix}$ and $\boldsymbol{b} = \begin{bmatrix}
    2 \\
    -5 \\
    h \\
\end{bmatrix}$.

\begin{align*}
\text{Augmented matrix: }
    \begin{bmatrix}
        1 & -3 & 2 \\
        -3 & 9 & -5 \\
        -5 & 15 & h \\
    \end{bmatrix}
\end{align*}
Row reducing:

\[
    R_3 \rightarrow R_3 + 5R_1 \text{ and } R_2 \rightarrow R_2 + 3R_1
\]

\[
    = \begin{bmatrix}
        1 & -3 & 2 \\
        0 & 0 & 1 \\
        0 & 0 & h + 10 \\
    \end{bmatrix}
\]

$\therefore$ From the echelon form above, the matrix equation, $A\boldsymbol{x} = \boldsymbol{b}$ is inconsistent ($0 \neq 1$ in $R_2$) no matter the value of $h$. Hence, $v_3$ is not in Span$\{v_1, v_2\}$.

\subsection*{b.}
To check for the value(s) of $h$ that makes the set $\{v_1, v_2, v_3\}$ linearly dependent, we need to check for the value(s) of $h$ that makes the matrix equation, $A\boldsymbol{x} = \boldsymbol{0}$ have a non-trivial solution. \\
\\
Where $A = \begin{bmatrix}
    v_1 & v_2 & v_3 \\
\end{bmatrix}
=\begin{bmatrix}
    1 & -3 & 2 \\
    -3 & 9 & -5 \\
    -5 & 15 & h \\
\end{bmatrix}$. \\
\\

Augmented matrix:
\[
    \begin{bmatrix}
        1 & -3 & 2 & 0 \\
        -3 & 9 & -5 & 0 \\
        -5 & 15 & h & 0 \\
    \end{bmatrix}
\]
Taking the echelon form of $A$ from part (a) above, we have:

\[
    \begin{bmatrix}
        1 & -3 & 2 & 0 \\
        0 & 0 & 1 & 0 \\
        0 & 0 & h + 10 & 0 \\
    \end{bmatrix}
\]
For the system above to be consistent, $h + 10 = 0 \implies h = -10$. Hence, the set $\{v_1, v_2, v_3\}$ is linearly dependent if and only if $h = -10$. The existence of a free variable, $x_2$ implies the existence of non-trivial solutions.

\section*{Problem 3}
\subsection*{a.}
\[
    T(x_1, x_2) = (2x_1-x_2, -3x_1+x_2, 2x_1-3x_2)
\]
\[
    A = \begin{bmatrix}
        2 & -1 \\
        -3 & 1 \\
        2 & -3 \\
    \end{bmatrix}
\]

\subsection*{b.}
\[
    T(\boldsymbol{x}) = (0,-1,-4)
\]

\[
    \begin{bmatrix}
        2 & -1 \\
        -3 & 1 \\
        2 & -3 \\
    \end{bmatrix} \begin{bmatrix}
        x_1 \\
        x_2 \\
    \end{bmatrix} = \begin{bmatrix}
        0 \\
        -1 \\
        -4 \\
    \end{bmatrix}
\]
Augumenting the matrix above and row reducing:
\[
    \begin{bmatrix}
        2 & -1 & 0 \\
        -3 & 1 & -1 \\
        2 & -3 & -4 \\
    \end{bmatrix}
\]

\[
    R_3 \rightarrow R_3 - 2R_1 \text{ and } R_2 \rightarrow R_2 + \frac{3}{2}R_1
\]

\[
    \begin{bmatrix}
        2 & -1 & 0 \\
        0 & -\frac{1}{2} & -1 \\
        0 & -2 & -4 \\
    \end{bmatrix}
\]

\[
    R_2 \rightarrow -2R_2 \text{ and } R_1 \rightarrow \frac{1}{2}R_1
\]

\[
    \begin{bmatrix}
        1 & -\frac{1}{2} & 0 \\
        0 & 1 & 2 \\
        0 & -2 & -4 \\
    \end{bmatrix}
\]

\[
    R_3 \rightarrow R_3 + 2R_2 \text{ and } R_1 \rightarrow R_1 + \frac{1}{2}R_2
\]

\[
    \begin{bmatrix}
        1 & 0 & 1 \\
        0 & 1 & 2 \\
        0 & 0 & 0 \\
    \end{bmatrix}
\]
From the reduced echelon form above, $x_1 = 1$ and $x_2 = 2$.

\[
    \therefore \boldsymbol{x} = \begin{bmatrix}
        1 \\
        2 \\
    \end{bmatrix}
\]

\subsection*{c.}
Row reduced echelon form of the standard matrix from part b:
\[
    \begin{bmatrix}
        2 & -1 \\
        -3 & 1 \\
        2 & -3 \\
    \end{bmatrix} \implies \begin{bmatrix}
        1 & 0 \\
        0 & 1 \\
        0 & 0 \\
    \end{bmatrix}
\]

There is a pivot position in every column. Hence, the columns of the standard matrix of T are linearly independent. Therefore, T is one-to-one. \\

There is not a pivot position in every row. Hence, the rows of the standard matrix of T are linearly dependent. Therefore, T is not onto.

\end{document}